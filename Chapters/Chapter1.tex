% Chapter 1

\chapter{Theory} % Main chapter title

\label{Chapter1} % For referencing the chapter elsewhere, use \ref{Chapter1} 

%----------------------------------------------------------------------------------------

% Define some commands to keep the formatting separated from the content 
\newcommand{\keyword}[1]{\textbf{#1}}
\newcommand{\tabhead}[1]{\textbf{#1}}
\newcommand{\code}[1]{\texttt{#1}}
\newcommand{\file}[1]{\texttt{\bfseries#1}}
\newcommand{\option}[1]{\texttt{\itshape#1}}

%----------------------------------------------------------------------------------------

\section{Atomar Fingerprints}
Atomar fingerprints can be used to measure similiarities and dissimilarities between different molecular structures and the atoms within. Quantizing the similarity between two structures just from measurement data is not a straight forward process, since the data only holds the cartesian coordinates of the atoms within the cell and the lattice vectors of said cell. These  quantities are not directly suitable for distinguishing different structures or the atoms within. The reasons are that the translation vectors are not unique and maybe transformed into linear combinations of themselves and describe the same structure. Furthermore, the coordinates and cell parameters can contain numerical noise and all the positions of the atoms maybe translated by an arbitrary vector as described by \cite{Lonie2012}. A fingerprint is then an abstract measure of characteristic for a structure that can be based on the afformentioned quantities but has the property of being easily comparable to other structures to test their similarity. The fingerprint discussed in this work is the one proposed by \cite{Zhu2016}. The environemnt of each atom is described by a fingerprint vector. This fingerprint vector can be calculated out of the experimental measurement data which includes the position of the atoms in the structure in cartesian coordinates and the lattice vectors of the structure. \\
\section{Calculation of the Fingerprint}
For each atom $k$ located at $\mathbf{R}_k$ in the crystal, we consider all neighbouring atoms contained in a sphere around $\mathbf{R}_k$. On each of those atoms, we put one or more Gaussian type orbitals, and calculate an overlap matrix as described by \cite{Sadeghi2013}.
\subsection{Overlap Matrix}
The overlap matrix of the GTO's has similar properties as the hamiltonian but is easily calculated analytically according to \cite{Sadeghi2013}. For this they introduce the normalized GTO's centered at the atomic positions $\mathbf{r}_i$ as
\begin{equation}\phi_i^l(\mathbf{r})=N_l(x-x_i)^{l_x}(y-y_j)^{l_y}(z-z_i)^{l_z}e^{-\alpha_i||\mathbf{r}-\mathbf{r}_i||^2}.\end{equation}
With $L=l_x+l_y+l_z$ as the angular momentum. The gaussian width $\alpha_i$ is proportional to the inverse of the covalent radius of atom $i$. In our work we modified the gaussian width $\alpha_i$ further to depend on the orbital eigenenergy of the corresponding orbitals. The orbitals considered were the 2s and 2p orbitals of Nitrogen and Carbon. The dependence on these energies was defined as:
\begin{equation}\tilde{\alpha}_i=C\cdot\frac{\alpha_i}{\left|E_{Orbital}\right|}\end{equation} 
The constant $C$ was then experimentaly determined. In the remainder of this work we will omit the tilde of $\tilde{\alpha}_i$.\\
The overlap integrals of two GTO's 
\begin{equation}\left<\phi_i^l\right|\left.\phi_j^{l'}\right>=\int d\mathbf{r}\,\phi_i^l(\mathbf{r})\phi^{l'}_j(\mathbf{r})\end{equation}
gives the normalization factors as
\begin{equation}N_l(\alpha_i)=\frac{1}{\sqrt{\left<\phi_i^l\right|\left.\phi_i^l\right>}}=\left(\frac{2\alpha_i}{\pi}\right)^{\frac{3}{4}}\sqrt{n_{l_x}n_{l_y}n_{l_z}}\end{equation}
with
$$n_k=\frac{(4\alpha_i)^k}{(2k-1)!!}$$
The Orbitals are then obtained by differentiating
$$\phi_i^s(\mathbf{r})=\left(\frac{2\alpha_i}{\pi}\right)^{\frac{3}{4}}e^{-\alpha_i||\mathbf{r}-\mathbf{r}_i||^2}$$
as
\begin{equation}\phi_i^{p_j}=\frac{1}{\sqrt{\alpha_i}}\frac{\partial\phi_i^s(\mathbf{r})}{\partial r_j}\end{equation}
The simplified relation for calculating the elements of the overlap matrix according to \cite{Clementi1966} are then
\begin{equation}S_{ij}=S_{ji}=\left(\frac{2\sqrt{\alpha_i\alpha_j}}{\alpha_i+\alpha_j}\right)^\frac{3}{2}\exp\left[\frac{-\alpha_i\alpha_j}{\alpha_i+\alpha_j}r_{ij}^2\right]\end{equation}
for the $s-s$ overlap. The $s-p$ overlap and the $p-p$ overlap is then obtained by
\begin{equation}\left<\phi_i^{p_x}\right|\left.\phi_j^s\right>=\\ \frac{1}{\sqrt{\alpha_i}}\frac{\partial S_{ij}}{\partial x_i}=-\left(\frac{2\sqrt{\alpha_i}\alpha_j}{\alpha_i+\alpha_j}\right)(x_i-x_j)S_{ij}\end{equation}
\begin{equation}\left<\phi_i^{p_x}\right|\left.\phi_j^{p_{x'}}\right>=\left(\frac{2\sqrt{\alpha_i\alpha_j}}{\alpha_i+\alpha_j}\right)S_{ij}\left[\delta_{x,x'}-\frac{2\alpha_i\alpha_j}{\alpha_i+\alpha_j}(x_i-x_j)(x'_i-x'_j)\right].\end{equation}
